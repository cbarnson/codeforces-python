\chapter{Precision issues and epsilons}

Computational geometry very often means working with floating-point values. Even when the input points are all integers, as soon as intermediate steps require things like line intersections, orthogonal projections or circle tangents, we have no choice but to use floating-point numbers to represent coordinates.

Using floating-point numbers comes at a cost: loss of precision. The number of distinct values that can be represented by a data type is limited by its number of bits, and therefore many ``simple'' values like $0.1$ or $\sqrt{2}$ cannot be exactly represented. Worse, even if $a$ and $b$ are exact, there is no guarantee that simple operations like $a+b$, $a-b$ or $ab$ will give an exact result.

Though many people are well aware that those issues exist, they will most often argue that they only cause small imprecisions in the answer in the end, and do not have any major consequence or the behavior of algorithms. In the rest of this chapter, we will show how both those assumptions can sometimes be false, then present some ways in which we \emph{can} make accurate statements about how precision loss affects algorithms, go on with a few practical examples, and finally give some general advice to problem solvers and setters.

\section{Small imprecisions can become big imprecisions}
In this section we explore two ways in which very small starting imprecisions can become very large imprecisions in the final output of a program.

\subsection{When doing numerically unstable computations}\label{ss:numerically-unstable}
There are some types of computations which can transform small imprecisions into catastrophically large ones, and line intersection is one of them. Imagine you have four points $A,B,C,D$ which were obtained through an previous imprecise process (for example, their position can vary by a distance of at most $r = 10^{-6}$), and you have to compute the intersection of lines $AB$ and $CD$.

For illustration, we represent the imprecisions on the points with small disk of radius $r$: the exact position is the black dot, while the small gray disk contains the positions it could take because of imprecisions. The dashed circle gives an idea of where point $I$ might lie.

In the best case, when $A,B$ and $C,D$ aren't too close together, and not too far from the intersection point $I$, then the imprecision on $I$ isn't too big.

\centerFig{small-big0}

But if those conditions are not respected, the intersection $I$ might vary in a very wide range or even fail to exist, if given the imprecision lines $AB$ and $CD$ end up being parallel, or if $A$ and $B$ (or $C$ and $D$) end up coinciding.

\centerFig{small-big1}
    
This shows that finding the intersection of two lines defined by imprecise points is a task that is inherently problematic for floating-point arithmetic, as it can produce wildly incorrect results even if the starting imprecision is quite small.

\subsection{With large values and accumulation}\label{ss:accumulation}

Another way in which small imprecisions can become big is by accumulation. Problem ``Keeping the Dogs Apart'', which we treat in more detail in a case study in section~\ref{ss:dogs}, is a very good example of this. In this problem, two dogs run along two polylines at equal speed and you have to find out the minimum distance between them at any point in time.

Even though the problem seems quite easy and the computations do not have anything dangerous for precision (mostly just additions, subtractions and distance computations), it turns out to be a huge precision trap, at least in the most direct implementation.

Let's say we maintain the current distance from the start for both dogs. There are $10^5$ polyline segments of length up to $\sqrt{2}\times 10^4$, so this distance can reach $\sqrt{2}\times 10^9$. Besides, to compute the sum, we perform $10^5$ sum operations which can all bring a $2^{-53} \approx 1.11 \times 10^{-16}$ relative error if we're using \lstinline|double|. So in fact the error might reach
\[\left(\sqrt{2}\times 10^9\right) \times 10^5 \times 2^{-53} \approx 0.016\]

Although this is a theoretical computation, the error does actually get quite close to this in practice, and since the tolerance on the answer is $10^{-4}$ this method actually gives a WA verdict.

This shows that even when only very small precision mistakes are made ($\approx 1.11 \times 10^{-16}$), the overal loss of precision can get very big, and carefully checking the maximal imprecision of your program is very important.

\section{Small imprecisions can break algorithms}
In this section, we explore ways in which small imprecisions can modify the behavior of an algorithm in ways other than just causing further imprecisions.

\subsection{When making binary decisions}
The first scenario we will explore is when we have to make clear-cut decisions, such as deciding if two objects touch.

{
    \newcommand{\eCutoff}{\epsilon_{\mathrm{cutoff}}}
    \newcommand{\eError}{\epsilon_{\mathrm{error}}}
    \newcommand{\eChance}{\epsilon_{\mathrm{chance}}}
    Let's say we have a line $l$ and a point $P$ computed imprecisely, and we want to figure out if the point lies on the line. Obviously, we cannot simply check if the point we have computed lies on the line, as it might be just slightly off due to imprecision. So the usual approach is to compute the distance from $P$ to $l$ and then figure out if that distance is less than some small value like $\eCutoff=10^{-9}$.

    While this approach tends to works pretty well in practice, to be sure that this solution works in every case and choose $\eCutoff$ properly,\footnote{And motivated problem setters \emph{do} tend to find the worst cases.} we need to know two things. First, we need to know $\eError$, the biggest imprecision that we might make while computing the distance. Secondly, and more critically, we need to know $\eChance$, the smallest distance that point $P$ might be from $l$ while not being on it, in other words, the closest distance that it might be from $l$ ``by coincidence''.

    Only once we have found those two values, and made sure that $\eError < \eChance$, can we then choose the value of $\eCutoff$ within $[\eError,\eChance)$.\footnote{In practice you should try to choose $\eCutoff$ so that there is a factor of safety on both sides, in case you made mistakes while computing $\eError$ or $\eChance$.} Indeed, if $\eCutoff < \eError$, there is a risk that $P$ is on $l$ but we say it is not, while if $\eCutoff \geq \eChance$ there is a risk that $P$ is not on $l$ but we say it is.

    Even though $\eError$ can be easily found with some basic knowledge of floating-point arithmetic and a few multiplications (see next section), finding $\eChance$ is often very difficult. It depends directly on which geometric operations were done to find $P$ (intersections, tangents, etc.), and in most cases where $\eChance$ can be estimated, it is in fact possible to make the comparison entirely with integers, which is of course the preferred solution.
}


\subsection{By violating basic assumptions}
Many algorithms rely on basic geometric axioms in order to provide their results, even though those assumptions are not always easy to track down. This is especially the case for incremental algorithms, like algorithms for building convex hulls. And when those assumptions are violated by using floating-point numbers, this can make algorithms break down in big ways.

Problems of this type typically happen in situation when points are very close together, or are nearly collinear/coplanar. The ways to solve the problem depend a lot on what the algorithm, but tricks like eliminating points that are too close together, or adding random noise to the coordinates to avoid collinearity/coplanarity can be very useful.

For concrete examples of robustness problems and a look into the weird small-scale behavior of some geometric functions, see \cite{classroom-robustness}.

\section{Modelling precision}
In this section, we try to build a basic model of precision errors that we can use to obtain a rough but reliable estimate of a program's precision.

\subsection{The issue with ``absolute or relative'' error}\label{ss:abs-rel}
When the output of a problem is some real value like a distance or an area, the problem statement often specifies a constraint such as: ``The answer should be accurate to an absolute or relative error of at most $10^{-5}$.'' While considering the relative accuracy of an answer can be a useful and convenient way to specify the required precision of an answer in some cases (for example in tasks where only addition and multiplication of positive values are performed), we think that for most geometry problems it is unsuitable.

The reason for this is the need to subtract\footnote{Strictly speaking, we mean both subtraction of values of the same sign, and addition of values of opposite signs.} large values of similar magnitude. For example, suppose that we are able to compute two values with relative precision $10^{-6}$, such as $A=1000 \pm 10^{-3}$ and $B = 999 \pm 10^{-3}$. If we compute their difference, we obtain $A-B = 1 \pm 2 \times 10^{-3}$. The absolute error remains of a comparable size, being only multiplied by 2, but on the other hand relative error increases drastically from $10^{-6}$ to $2\times 10^{-3}$ because of the decrease in magnitude. This phenomenon is called \emph{catastrophic cancellation}.

In fact, whenever a certain relative error can affect big numbers, catastrophic cancellation can cause the corresponding absolute error to appear on very small values. The consequence is that if a problem statement has a certain tolerance on the relative error of the answer, and a correct solution has an error close to it for the biggest possible values, then the problem statement also needs to specify a tolerance on the corresponding absolute error in case catastrophic cancellation happens. And since that tolerance on absolute error is at least as tolerant as the tolerance on relative error for all possible values, it makes it redundant. This is why we think that tolerance on ``absolute or relative error'' is misleading at best.\footnote{In fact, working with relative error tolerances would make sense if this ``relative error'' was defined based on the magnitude of the input coordinates rather than on the magnitude of the answer, as we will see starting from section~\ref{ss:biggest-mag}. For example, if all input coordinates are bounded by $M$, it would make sense to require an absolute precision of $M^2\times 10^{-6}$ on an area. But since the answer can remain very small even if the magnitude of the input grows, requiring a fixed relative precision on it is usually too constraining for test cases with inputs of large magnitude.}

Catastrophic cancellation shows that relative precision is not a reliable way to think about precision whenever subtractions are involved --- and that includes the wide majority of geometry problems.
In fact, the most common geometric operations (distances, intersections, even dot/cross products) all involve subtractions of values which could be very similar in magnitude.

Examples of this appear in two of the case studies of section~\ref{s:case-studies}: in problem ``Keeping the Dogs Apart'' and when finding the solution of a quadratic equation.

Another example occurs when computing areas of polygons made of imprecise points. Even when the area ends up being small, the imprecision on it can be large if there were computations on large values in intermediate steps, which is the case when the coordinates have large magnitudes.

\centerFig{modelling0}

Because of this, we advise against trying to use relative error to build precision guarantees on the global scale of a whole algorithm, and we recommend to reason about those based on absolute error instead, as we describe below.

\subsection{Precision guarantees from IEEE 754}
%\emph{Note: This section is not meant to be a full description of how floating-point numbers work, but only a reminder of some useful guarantees. If you are completely unfamiliar with how floating-point numbers work or want to know more details, a good reference is \cite{goldberg-floating}.}

Nearly all implementations of floating-point numbers obey the specifications of the IEEE 754 standard. This includes \lstinline|float| and \lstinline|double| in Java and C++, and \lstinline|long double| in C++. The IEEE 754 standard gives strong guarantees that ensure floating-point numbers will have similar behavior even in different languages and over different platforms, and gives users a basis to build guarantees on the precision of their computations.

The basic guarantees given by the standard are:
\begin{enumerate}
\item decimal values entered in the source code or a file input are represented by the closest representable value;
\item the five basic operations ($+,-,\times,/,\sqrt{x}$) are performed as if they were performed with infinite precision and then rounded to the closest representable value.
\end{enumerate}

There are several implications. First, this means that integers are represented exactly, and basic operations on them ($+,-,\times$) will have exact results, as long as they are small enough to fit within the significant digits of the type: $\geq 9\times 10^{15}$ for \lstinline|double|, and $\geq 1.8 \times 10^{19}$ for \lstinline|long double|. In particular, \lstinline|long double| can perform exactly all the operations that a 64-bit integer type can perform.

Secondly, if the inputs are exact, the relative error on the result of any of those five operations ($+,-,\times,/,\sqrt{x}$) will be bounded by a small constant that depends on the number of significant digits in the type.\footnote{This assumes the magnitudes do not go outside the allowable range ($\approx 10^{\pm 308}$ for \lstinline|double| and $\approx 10^{\pm 4932}$ for \lstinline|long double|) which almost never happens for geometry problems.} This constant is $< 1.2 \times 10^{-16}$ for \lstinline|double| and $< 5.5 \times 10^{-20}$ for \lstinline|long double|. It is called the \emph{machine epsilon} and we will often write it $\epsilon$.

\subsection{Considering the biggest possible magnitude}\label{ss:biggest-mag}
We explained earlier why we need to work with absolute error, but since IEEE 754 gives us guarantees in terms of relative errors, we need to consider the biggest magnitude that will be reached during the computations. In other words, if all computations are precise up to a relative error of $\epsilon$, and the magnitude of the values never goes over $M$, then the absolute error of an operation is at most $M\epsilon$.

This allows us to give good guarantees for numbers obtained after a certain number of $+$ and $-$ operations: a value that is computed in $n$ operations\footnote{Note that when we say a value is ``computed in $n$ operations'' we mean that it is computed by a single formula that contains $n$ operations, and not that $n$ operations are necessary to actually compute it. For example $(a+b)+(a+b)$ is considered to be ``computed in 3 operations'' even though we can implement this with only 2 additions.} will have an absolute error of at most $nM\epsilon$ compared to the theoretical result.
%\footnote{By this we mean, a value that is computed by a single formula involving only $+$ and $-$ operations, and at most $n$ of them.}

%To clarify what we mean here by ``computed in $n$ operations'', we define it this way:
%\begin{itemize}
%\item exact inputs are ``computed in $0$ operations'';
%\item if $a$ is ``computed in $n_a$ operations'' and $b$ is ``computed in $n_b$ operations'' then the sum $a+b$ and subtraction $a-b$, is ``computed in $n_a+n_b+1$ operations''.
%\end{itemize}



We can prove the guarantee by induction: let's imagine we have two intermediate results $a$ and $b$ who were computed in $n_a$ and $n_b$ operations respectively. By the inductive hypothesis their imprecise computed values $a'$ and $b'$ respect the following conditions.
\[|a'-a| \leq n_aM\epsilon \qquad |b'-b| \leq n_bM\epsilon\]

The result of the floating-point addition of $a'$ and $b'$ is $\round(a'+b')$ where $\round()$ is the function that rounds a real value to the closest representable floating-point value. We know that $|\round(x)-x| \leq M\epsilon$, so we can find a bound on the error of the addition:
\begin{align*}
|\round(a'+b') &- (a+b)|\\
&= \left|\left[\round(a'+b') - (a'+b')\right] + \left[(a'+b') - (a+b)\right]\right|\\
&\leq |\round(a'+b') - (a'+b')| + |(a'+b') - (a+b)|\\
&\leq M\epsilon + |(a'-a) + (b'-b)| \\
&\leq M\epsilon + |a'-a| + |b'-b| \\
&\leq M\epsilon + n_aM\epsilon + n_bM\epsilon \\
&= (n_a + n_b + 1) M \epsilon
\end{align*}
where the first two steps follow from the triangle inequality.
Since the sum is ``computed in $n_a+n_b+1$ operations'', the bound of $(n_a + n_b + 1) M \epsilon$ that is obtained is small enough. The proof for subtraction is very similar.

\subsection{Incorporating multiplication}
The model above gives good guarantees but is very limited: it only works for computations that use only addition and subtraction. Multiplication does not give guarantees of the form $nM\epsilon$. However, we can still say interesting things if we take a closer look the different types of values we use in geometry:
\begin{itemize}
\item Adimensional ``0D'' values: e.g. angles, constant factors;% Their magnitude is typically limited by some small constant (e.g. $2\pi$ for angles).
\item 1D values: e.g. coordinates, lengths, distances, radii;% Their magnitude is limited by a constant $M$, which can be deduced from the input limits.
\item 2D values: e.g. areas, dot products, cross products;% Since they are based on products of 1D numbers, their magnitude is typically within a small constant factor of $M^2$.
\item 3D values: e.g. volumes, mixed products.% For the same reasons, their magnitude is typically within a small constant factor of $M^3$.
\end{itemize}

Usually, the problem statement gives guarantees on the magnitude of coordinates, so we can find some constant $M$ so that all 1D values that will be computed in the code have a magnitude less than $M$. And since 2D and 3D values are usually created by products of 1D values, we can usually say that 2D values are bounded in magnitude by $M^2$ and 3D values by $M^3$ (we may need to multiply $M$ by a constant factor).

It turns out that computations made of $+,-,\times$ and in which all $d$-dimensional values are bounded in magnitude by $M^d$ have good precision guarantees. In fact, we can prove that the absolute error of a $d$-dimensional number computed in $n$ operations is at most $M^d\left((1+\epsilon)^n-1\right)$, which assuming $n\epsilon \ll 1$ is about $nM^d\epsilon$.

The proof is similar in spirit to what we did with only $+$ and $-$ earlier. Since it is a bit long, we will not detail it here, but it can be found in section~\ref{sec:proof-precision}, along with a more precise definition of the precision guarantees and its underlying assumptions.

Note that this does \emph{not} cover multiplication by an adimensional factor bigger than 1: this makes sense, since for example successive multiplication by 2 of a small value could make the absolute error grow out of control even if the magnitude remains under $M^d$ for a while.

In other cases, this formula $nM^d\epsilon$ gives us a quick and reliable way to estimate precision errors.

\subsection{Why other operations do not work as well}\label{ss:other-operations}
Now that we have precision guarantees for $+,-,\times$ operations, one might be tempted to try and include division as well. However, if that was possible, then it would be possible to give strong precision guarantees for line intersection, and we saw in subsection~\ref{ss:numerically-unstable} that this is not the case.

The core of the problem is: if some value $x$ is very close to zero, then a small absolute error on $x$ will create a large absolute error on $1/x$. In fact, if $x$ is smaller than its absolute error, the computed value $1/x$ might be arbitrarily big, both in the positive or negative direction, and might not exist. This is why it is hard to give guarantees on the results of a division whose operands are already imprecise.

An operation that also has some problematic behavior is $\sqrt{x}$. If $x$ is smaller than its absolute error, then $\sqrt{x}$ might or might not be defined in the reals. However, if we ignore the issue of existence by assuming that the theoretical and actual value of $x$ are both nonnegative, then we \emph{can} say some things on the precision.

Because $\sqrt{x}$ is a concave increasing function, a small imprecision on $x$ will have the most impact on $\sqrt{x}$ near 0.

\centerFig{modelling1}

Therefore for a given imprecision $\delta$, the biggest imprecision on $\sqrt{x}$ it might cause is $\sqrt{\delta}$. This is usually pretty bad: if the argument of the square root had an imprecision of $nM^2\epsilon$ then in the worst case the result will have an imprecision of $\sqrt{n}M\sqrt{\epsilon}$, instead of the $nM\epsilon$ bound that we have for $+,-,\times$ operations.

For example let us consider a circle $\mathcal{C}$ of radius tangent to a line $l$. If $\mathcal{C}$ gets closer to $l$ by $10^{-6}$, then the intersection points will move by about
\[\sqrt{1^2 - (1-10^{-6})^2} \approx \sqrt{2\times 10^{-6}} = \sqrt{2} \times 10^{-3}\]
away from the tangency point, as pictured below.

\centerFig{modelling2}

Note that here we have only shown that $1/x$ and $\sqrt{x}$ perform poorly on imprecise inputs. Please bear in mind that on exact inputs, the IEEE 754 guarantees that the result is the closest represented floating-point number. So when the lines and circles are defined by integers, line intersections and circle-line intersections have a relative precision error proportional to $\epsilon$ and thus an absolute error proportional to $M\epsilon$.

%\subsection{Notations}
%\todo{}
\section{Case studies}\label{s:case-studies}
In this section, we explore some practical cases in which the imprecisions of floating-point numbers can cause problems and give some possible solutions.

\subsection{Problem ``Keeping the Dogs Apart''}\label{ss:dogs}
We will first talk about problem ``Keeping the Dogs Apart'', which we mentioned before, because it is a good example of accumulation of error and how to deal with it. It was written by Markus Fanebust Dregi for NCPC 2016. You can read the full statement and submit it at \url{https://open.kattis.com/problems/dogs}.

Here is a summarized problem statement: There are two dogs A and B, walking at the same speed along different polylines $A_0 \ldots A_{n-1}$ and $B_0 \ldots B_{m-1}$, made of 2D integer points with coordinates in $[0,10^4]$. They start at the same time from $A_0$ and $B_0$ respectively. What is the closest distance they will ever be from each other before one of them reaches the end of its polyline? The relative/absolute error tolerance is $10^{-4}$, and $n,m \leq 10^5$.

The idea of the solution is to divide the time into intervals where both dogs stay on a single segment of their polyline. Then the problem reduces to the simpler task of finding the closest distance that get when one walks on $[PQ]$ and the other on $[RS]$, with $|PQ|=|RS|$. This division into time intervals can be done with the two-pointers technique: if we remember for each dog how many segments it has completely walked and their combined length, we can work out when is the next time on of the dogs will switch segments.

The main part of the code looks like this. We assume that \lstinline|moveBy(a,b,t)| gives the point on a segment $[AB]$ at a certain distance $t$ from $A$, while \lstinline|minDist(p,q,r,s)| gives the minimum distance described above for $P,Q,R,S$.
\begin{lstlisting}
int i = 0, j = 0; // current segment of A and B
double ans = abs(a[0]-b[0]), // closest distance so far
      sumA = 0, // total length of segments fully walked by A
      sumB = 0; // total length of segments fully walked by B

// While both dogs are still walking
while (i+1 < n && j+1 < m) {
    double start = max(sumA, sumB), // start of time interval
              dA = abs(a[i+1]-a[i]), // length of current segment of A
              dB = abs(b[j+1]-b[j]), // length of current segment of B
            endA = sumA + dA, // time at which A will end this segment
            endB = sumB + dB, // time at which B will end this segment
             end = min(endA, endB); // end of time interval
    
    // Compute start and end positions of both dogs
    pt p = moveBy(a[i], a[i+1], start-sumA),
       q = moveBy(a[i], a[i+1], end-sumA),
       r = moveBy(b[j], b[j+1], start-sumB),
       s = moveBy(b[j], b[j+1], end-sumB);
    
    // Compute closest distance for this time interval
    ans = min(ans, minDist(p,q,r,s));
    
    // We get to the end of the segment for one dog or the other,
    // so move to the next and update the sum of lengths
    if (endA < endB) {
        i++;
        sumA += dA;
    } else {
        j++;
        sumB += dB;
    }
}
// output ans
\end{lstlisting}

As we said in section~\ref{ss:accumulation}, the sums \lstinline|sumA| and \lstinline|sumB| accumulate very large errors. Indeed, they can both theoretically reach $M=\sqrt{2}\times 10^9$, and are based on up to $k=10^5$ additions. With \lstinline|double|, $\epsilon = 2^{-53}$, so we could reach up to $kM\epsilon \approx 0.016$ in absolute error in both \lstinline|sumA| and \lstinline|sumB|. Since this error directly translates into errors in $P,Q,R,S$ and is bigger than the tolerance of $10^{-4}$, this causes WA.

In the rest of this section, we will look at two ways we can avoid this large accumulation of error in \lstinline|sumA| and \lstinline|sumB|. Since this is currently much bigger than what could have been caused by the initial length computations, \lstinline|moveBy()| and \lstinline|minDist()|, we will consider those errors to be negligible for the rest of the discussion.

\subsubsection{Limiting the magnitude involved}
The first way we can limit the accumulation of error in \lstinline|sumA| and \lstinline|sumB| is to realize that in fact, we only care about the difference between them: if we add a certain constant to both variables, this doesn't change the value of \lstinline|start-sumA|, \lstinline|end-sumA|, \lstinline|start-sumB| or \lstinline|end-sumB|, so the value of \lstinline|p|, \lstinline|q|, \lstinline|r|, \lstinline|s| is unchanged.

So we can adapt the code by adding these lines at the end of the \lstinline|while| loop:
\begin{lstlisting}
double minSum = min(sumA, sumB);
sumA -= minSum;
sumB -= minSum;
\end{lstlisting}

After this, one of \lstinline|sumA| and \lstinline|sumB| becomes zero, while the other carries the error on both. In total, at most $n+m$ additions and $n+m$ subtractions are performed on them, for a total of $k \leq 4\times 10^5$. But since the difference between \lstinline|sumA| and \lstinline|sumB| never exceeds the length of one segment, that is, $M=\sqrt{2} \times 10^4$, the error is much lower than before:
\[kM\epsilon = \left(4\times 10^5\right) \times \left(\sqrt{2}\times 10^4\right) \times 2^{-53} \approx 6.3 \times 10^{-7}\]
so it gives an AC verdict.

So here we managed to reduce the precision mistakes on our results by reducing the magnitude of the numbers that we manipulate. Of course, this is only possible if the problem allows it.

\subsubsection{Summing positive numbers more precisely}
Now we present different way to reduce the precision mistake, based on the fact that all the terms in the sum we're considering are positive. This is a good thing, because it avoids catastrophic cancellation (see section~\ref{ss:abs-rel}).

In fact, addition of nonnegative numbers conserves relative precision: if you sum two nonnegative numbers $a$ and $b$ with relative errors of $k_a\epsilon$ and $k_b\epsilon$ respectively, the worst-case relative error on $a+b$ is about\footnote{It could in fact go up to $(\max(k_a,k_b)(1+\epsilon)+1)\epsilon$ but the difference is negligible for our purposes.} $(\max(k_a,k_b)+1)\epsilon$.

Let's say we need to compute the sum of $n$ nonnegative numbers $a_1,\ldots,a_n$. We suppose they are exact. If we perform the addition in the conventional order, like this:
\[\Bigg(\cdots\Big(\big((a_1+a_2)+a_3\big)+a_4\Big)+\cdots\Bigg)+a_n\]
then
\begin{itemize}
\item $a_1+a_2$ will have a relative error of $(\max(0,0)+1)\epsilon = \epsilon$;
\item $(a_1+a_2)+a_3$ will have a relative error of $(\max(1,0)+1)\epsilon = 2\epsilon$;
\item $\big((a_1+a_2)+a_3\big)+a_4$ will have a relative error of $(\max(2,0)+1)\epsilon = 3\epsilon$;
\item \ldots and so on.
\end{itemize}
So the complete sum will have an error of $(n-1)\epsilon$, not better than what we had before.

But what if we computed the additions in another order? For example, with $n=8$, we could do this:
\[\big((a_1+a_2)+(a_3+a_4)\big) + \big((a_5+a_6)+(a_7+a_8)\big)\]
then all additions of two numbers have error $\epsilon$, all additions of 4 numbers have error $(\max(1,1)+1)\epsilon = 2\epsilon$, and the complete addition has error $(\max(2,2)+1)\epsilon = 3\epsilon$, which is much better than $(n-1)\epsilon = 7\epsilon$. In general, for $n = 2^k$, we can reach a relative precision of $k\epsilon$.

We can use this grouping technique to create an accumulator such that the relative error after adding $n$ numbers is at most $2\log_2(n)\epsilon$.\footnote{We could even get $(\log_2{n}+1)\epsilon$ but we don't know a way to do it faster than $O(n\log n)$.} Here is an $O(n)$ implementation:
\begin{lstlisting}
struct stableSum {
    int cnt = 0;
    vector<double> v, pref{0};
    void operator+=(double a) {
        assert(a >= 0);
        int s = ++cnt;
        while (s % 2 == 0) {
            a += v.back();
            v.pop_back(), pref.pop_back();
            s /= 2;
        }
        v.push_back(a);
        pref.push_back(pref.back() + a);
    }
    double val() {return pref.back();}
};
\end{lstlisting}

Let's break this code down. This structure provides two methods: \lstinline|add(a)| to add a number $a$ to the sum, and \lstinline|val()| to get the current value of the sum. Array \lstinline|v| contains the segment sums that currently form the complete sum, similar to Fenwick trees: for example, if we have added 11 elements, \lstinline|v| would contain three elements:
\[v = \{a_1+\cdots+a_8,\ a_9+a_{10},\ a_{11}\}\]
while \lstinline|pref| contains the prefix sums of \lstinline|v|: \lstinline|pref[i]| contains the sum of the $i$ first elements of \lstinline|v|.

Function \lstinline|add()| performs the grouping: when adding a new element \lstinline|a|, it will merge it with the last element of \lstinline|v| while they contain the same number of terms, then \lstinline|a| is added to the end of \lstinline|v|. For example, if we add the $12\nth$ element $a_{12}$, the following steps will happen:
\begin{align*}
v &= \{a_1+\cdots+a_8,\ a_9+a_{10},\ a_{11}\} \quad &a = a_{12}\\
v &= \{a_1+\cdots+a_8,\ a_9+a_{10}\} \quad &a = (a_{11})+a_{12}\\
v &= \{a_1+\cdots+a_8\} \quad &a = (a_9+a_{10})+a_{11}+a_{12}\\
v &= \{a_1+\cdots+a_8,\ a_9+a_{10}+a_{11}+a_{12}\}
\end{align*}

The number of additions we have to make for the $i\nth$ number is the number of times it is divisible by 2. Since we only add one element to \lstinline|v| when adding an element to the sum, this is amortized constant time.

By simply changing the types of \lstinline|sumA| and \lstinline|sumB| to \lstinline|stableSum| and adding \lstinline|.val()| whenever the value is read in the initial code, we can get down to an error of about
\[2\log_2\big(10^5\big) M \epsilon = \Big(2\log_2\big(10^5\big)\Big) \times \left(\sqrt{2} \times 10^9\right) \times 2^{-53} \approx 5.2 \times 10^{-6}\]
which also gives an AC verdict.\footnote{Theoretically we can't really be sure though, since both \lstinline|sumA| and \lstinline|sumB| could have that error, and we still have to take into account the other operations performed.}

This is not as good as the precision obtained with the previous method, but that method was specific to the problem, while this one can be applied whenever we need to compute sums of nonnegative numbers.

\subsection{Quadratic equation}
As another example, we will study the precision problems that can occur when computing the roots of an equation of the type $ax^2+bx+c=0$ with $a\neq 0$. We will see how some precision problems are unavoidable, while others can be circumvented by 

When working with full-precision reals, we can solve quadratic equations in the following way. First we compute the discriminant $\Delta = b^2-4ac$. If $\Delta < 0$, there is no solution, while if $\Delta \geq 0$ there is are 1 or 2 solutions, given by \[x = \frac{-b \pm \sqrt{\Delta}}{2a}\]

The first difficulty when working with floating-point numbers is the computation of $\Delta$: if $\Delta \approx 0$, that is $b^2 \approx 4ac$, then the imprecisions can change the sign of $\Delta$, therefore changing the number of solutions.

Even if that does not happen, since we have to perform a square root, the problems that we illustrated with line-circle intersection in section~\ref{ss:other-operations} can also happen here.\footnote{Which is not surprising, since the bottom of a parabola looks a lot like a circle.} Take the example of equation $x^2-2x+1=0$, which is a single root $x=0$. If there is a small error on $c$, it can translate into a large error on the roots. For example, if $c=1-10^{-6}$, then the roots become
\[x = \frac{-b \pm \sqrt{b^2-4ac}}{2a} = \frac{2 \pm \sqrt{4 - 4(1-10^{-6})}}{2} = \frac{2 \pm 2\sqrt{10^{-6}}}{2} = 1 \pm 10^{-3}.\]
where the error $10^{-3}$ is much bigger than the initial error on $c$.

\centerFig{cases0}

Even if the computation of $\sqrt{\Delta}$ is very precise, a second problem can occur. If $b$ and $\sqrt{\Delta}$ have a similar magnitude, in other words when $b^2 \gg ac$, then catastrophic cancellation will occur for one of the roots. For example if $a=1,b=10^4,c=1$, then the roots will be:
\[x_1 = \frac{-10^4 - \sqrt{10^8-4}}{2} \approx -10^4 \qquad x_2 = \frac{-10^4 + \sqrt{10^8-4}}{2} \approx 10^{-4}\]

The computation of $x_1$ goes fine because $-b$ and $-\sqrt{\Delta}$ have the same sign. But because the magnitude of $-b+\sqrt{\Delta}$ is $10^8$ times smaller than the magnitude of $b$ and $\sqrt{\Delta}$, the relative error on $x_2$ will be $10^8$ times bigger than the relative error on $b$ and $\sqrt{\Delta}$.

Fortunately, in this case we can avoid catastrophic cancellation entirely by rearranging the expression:
\begin{align*}
\frac{-b \pm \sqrt{b^2-4ac}}{2a}
&= \frac{-b \pm \sqrt{b^2-4ac}}{2a} \times \frac{-b \mp \sqrt{b^2-4ac}}{-b \mp \sqrt{b^2-4ac}}\\
&= \frac{(-b^2) - \left(\sqrt{b^2-4ac}\right)^2}{2a \left(-b \mp \sqrt{b^2-4ac}\right)}\\
&= \frac{4ac}{2a \left(-b \mp \sqrt{b^2-4ac}\right)}\\
&= \frac{2c}{-b \mp \sqrt{b^2-4ac}}
\end{align*}
In this new expression, since the sign of the operation is opposite from the sign in the original expression, catastrophic cancellation happens in only one of the two.

So if $b \geq 0$, we can use $\frac{-b-\sqrt{\Delta}}{2a}$ for the first solution and $\frac{2c}{-b-\sqrt{\Delta}}$ for the second solution, while if $b \leq 0$, we can use $\frac{2c}{-b+\sqrt{\Delta}}$ for the first solution and $\frac{-b+\sqrt{\Delta}}{2a}$ for the second solution. We only need to be careful that the denominator  is never zero.

This gives a safer way to find the roots of a quadratic equation. This function returns the number of solutions, and places them in \lstinline|out| in no particular order.
\begin{lstlisting}
int quadRoots(double a, double b, double c, pair<double,double> &out) {
    assert(a != 0);
    double disc = b*b - 4*a*c;
    if (disc < 0) return 0;
    double sum = (b >= 0) ? -b-sqrt(disc) : -b+sqrt(disc);
    out = {sum/(2*a), sum == 0 ? 0 : (2*c)/sum};
    return 1 + (disc > 0);
}
\end{lstlisting}

In many cases, there are several ways to write an expression, and they can have very different behaviors when used with floating-point numbers. So if you realize that the expression you are using can cause precision problems in some cases, it can be a good idea to rearrange the expression to handle them, as we did here.

\subsection{Circle-circle intersection}
This last case study will study one possible implementation for the intersection of two circles. It will show us why we shouldn't rely too much on mathematical truths when building our programs.

We want to know whether two circles of centers $C_1,C_2$ and radii $r_1,r_2$ touch, and if they do what are the intersection points. Here, we will solve this problem with triangle inequalities and the cosine rule.\footnote{The way we actually implement it in this book is completely different.} Let $d = |C_1C_2|$. The question of whether the circles touch is equivalent to the question of whether there exists a (possibly degenerate) triangle with edge lengths $d,r_1,r_2$.

\centerFig{cases1}

We know that such a triangle exists iff the triangle inequalities are respected, that is:
\[|r_2-r_1| \leq d \leq r_1+r_2\]
If this is true, then we can find the angle at $C_1$, which we'll call $\alpha$, thanks to the cosine rule:
\[\cos\alpha = \frac{d^2+r_1^2-r_2^2}{2dr_1}\]

Once we have $\alpha$, we can find the intersection points in the following way: if we take vector $\vv{C_1C_2}$, resize it to have length $r_1$, then rotate by $\alpha$ in either direction, this gives the vectors from $C_1$ to either intersection points.

\centerFig{cases2}

This gives the following code. It uses a function \lstinline|abs()| to compute the length of a vector (see section~\ref{ss:point-representation}) and a function \lstinline|rot()| to rotate a vector by a given angle (see section~\ref{ss:rotation}).
\begin{lstlisting}
bool circleCircle(pt c1, double r1, pt c2, double r2, pair<pt,pt> &out) {
    double d = abs(c2-c1);
    if (d < abs(r2-r1) || d > r1+r2) // triangle inequalities
        return false;
    double alpha = acos((d*d + r1*r1 - r2*r2)/(2*d*r1));
    pt rad = (c2-c1)/d*r1; // vector C1C2 resized to have length d
    out = {c1 + rot(rad, -alpha), c1 + rot(rad, alpha)};
    return true;
}
\end{lstlisting}

This implementation is quite nice, but unfortunately it will sometimes output \lstinline|nan| values. In particular, if
\[r_1 = 0.625 \qquad r_2 = 0.3750000000000004 \qquad d = 1.0000000000000004\]
then the triangle inequalities are respected, so the function returns \lstinline|true|, but the program computes
\[\frac{d^2 + r_1^2 - r_2^2}{2dr_1} > 1\]

In fact, this is mathematically impossible! The cosine rule should give values in $[-1,1]$ as long as the edge lengths respect the triangle inequality. To make sure, we can compute:
\begin{align*}
\frac{d^2 + r_1^2 - r_2^2}{2dr_1} > 1\ 
&\Rightarrow\ d^2 + r_1^2 - r_2^2 > 2dr_1\\
&\Leftrightarrow\ (d-r_1)^2 > r_2^2\\
&\Leftrightarrow\ |d-r_1| > r_2\\
&\Leftrightarrow\ d > r_2+r_1 \quad\mbox{or}\quad r_1 > d+r_2
\end{align*}
Indeed, both are impossible because of the triangle inequalities. So this must be the result of a few unfortunate roundings made while computing the expression.

There are two possible solutions to this. The first solution would be to just treat the symptoms: make sure the cosine is never outside $[-1,1]$ by either returning \lstinline|false| or by moving it inside:
\begin{lstlisting}
double co = (d*d + r1*r1 - r2*r2)/(2*d*r1);
if (abs(co) > 1) {
    return false; // option 1
    co /= abs(co); // option 2
}
double alpha = cos(co);
\end{lstlisting}

The second solution, which we recommend, is based on the principles that we should always try to minimize the number of comparisons we make, and that if we have to do some computation that might fail (giving a result of \lstinline|nan| or infinity), then we should test the input of that computation \emph{directly}.

So instead of testing the triangle inequalities, we test the value of $\cos\alpha$ directly, because it turns out that it will be in $[-1,1]$ iff the triangle inequalities are verified. This gives the following code, which is a bit simpler and safer.
\begin{lstlisting}
bool circleCircle(pt c1, double r1, pt c2, double r2, pair<pt,pt> &out) {
    double d = abs(c2-c1), co = (d*d + r1*r1 - r2*r2)/(2*d*r1);
    if (abs(co) > 1) return false;
    double alpha = acos(co);
    pt rad = (c2-c1)/d*r1; // vector C1C2 resized to have length d
    out = {c1 + rot(rad, -alpha), c1 + rot(rad, alpha)};
    return true;
}
\end{lstlisting}

\section{Some advice}
In this last section, we present some general advice about precision issues when solving or setting a problem.

\subsection{For problem solvers}
One of the keys to success in geometry problems is to develop a reliable implementation methodology as you practise. Here are some basics to get you started.

As you have seen in this chapter, using floating-point numbers can cause many problems and betray you in countless ways. Therefore the first and most important piece of advice is to avoid using them altogether. Surprisingly many geometric computations can be done with integers, and you should always aim to perform important comparisons with integers, by first figuring out the formula on paper and then implementing it without division or square root.

When you are forced to use floating-point numbers, you should minimize the risks you take. Indeed, thinking about everything that could go wrong in an algorithm is very hard and tedious, so if you take many inconsiderate risks, the time you will need to spend too much time on verification (or not spend it and suffer the consequences). In particular:
\begin{itemize}
\item Minimize the number of dangerous operations you make, such as divisions, square roots, and trigonometric functions. Some of these functions can amplify precision mistakes, and many are defined on restricted domains. Make sure you do not go out of the domains by considering every single one of them carefully.
\item Separate cases sparingly. Many geometry problems require some casework, making comparisons to separate them can be unsafe, and every case adds more code and more reasons for failures. When possible, try to write code that handles many situations at once.
\item Do not rely too much on mathematical truths. Things that are true for reals are not necessarily true for floating-point numbers. For example, $r^2-d^2$ and $(r+d)(r-d)$ are not always exactly the same value. Be extra careful when those values are then used in an operation that is not defined everywhere (like $\sqrt{x}$, $\arccos(x)$, $\tan(x)$, $\frac{x}{y}$ etc.).
\end{itemize}

In general, try to build programs that are resistant to the oddities of floating-point numbers. Imagine that some evil demon is slightly modifying every result you compute in the way that is most likely to make your program fail. And try to write clean code that is \emph{clearly correct} at first glance. If you need long explanations to justify why your program will not fail, then it is more likely that your program will in fact fail.

\subsection{For problem setters}
Finally, here is some general advice about precision issues when creating a geometry problem and its datasets.
\begin{itemize}
\item Never use floating-point numbers as inputs, as this will already cause imprecisions when first reading the input numbers, and completely exclude the use of integers make it impossible to determine some things with certainty, like whether two segmentns touch, whether some points are collinear, etc.
\item Make the magnitude of the input coordinates as small as possible to avoid causing overflows or big imprecisions in the contestant's codes.
\item Favor problems where the important comparisons can be made entirely with integers.
\item Avoid situations in which imprecise points are used for numerically unstable operations such as finding the intersection of two lines.
\item In most cases, you should specify the tolerance in terms of absolute error only (see subsection~\ref{ss:abs-rel}).
\item Make sure to prove that all correct algorithm are able to reach the precision that you require, and be careful about operations like circle-line intersection which can greatly amplify imprecisions. Since error analysis is more complicated than it seems at first sight and requires a bit of expertise, you may want to ask a friend for a second opinion.
\end{itemize}

