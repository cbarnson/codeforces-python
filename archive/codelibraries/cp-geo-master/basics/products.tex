\section{Products and angles}
Besides complex multiplication, which is nice to have but is not useful so often, there are two products involving vectors that are of critical importance: dot product and cross product. In this section, we'll look at their definition, properties and some basic use cases.

\subsection{Dot product}\label{ss:dot}
The dot product $\dotv{v}{w}$ of two vectors $\vv{v}$ and $\vv{w}$ can be seen as a measure of how similar their directions are. It is defined as \[\dotv{v}{w} = \normv{v}\normv{w} \cos\theta\] where $\normv{v}$ and $\normv{w}$ are the lengths of the vectors and $\theta$ is amplitude of the angle between $\vv{v}$ and $\vv{w}$.

Since $\cos(-\theta) = \cos(\theta)$, the sign of the angle does not matter, and the dot product is symmetric: $\dotv{v}{w} = \dotv{w}{v}$.

In general we will take $\theta$ in $[0,\pi]$, so that dot product is positive if $\theta < \pi/2$, negative if $\theta > \pi/2$, and zero if $\theta = \pi/2$, that is, if $\vv{v}$ and $\vv{w}$ are perpendicular.
\centerFig{products0}

If we fix $\normv{v}$ and $\normv{w}$ as above, the dot product is maximal when the vectors point in the same direction, because $\cos\theta = \cos(0) = 1$, and minimal when they point in opposite directions, because $\cos\theta = \cos(\pi) = -1$.

\begin{mathy}
Because of the definition of cosine in right triangles, dot product can be interpreted in an interesting way (assuming $\vv{v} \neq 0$): $\vv{v}\cdot\vv{w} = \normv{v}\projv{v}{w}$ where $\projv{v}{w} \coloneqq \normv{w}\cos\theta$ is the signed length of the projection of $\vv{w}$ onto the line that contains $\vv{v}$ (see examples below).
In particular, this means that the dot product does not change if one of the vectors moves perpendicular to the other.
\centerFig{products1}
\end{mathy}

Remarkably, the dot product can be computed by a very simple expression: if $\vv{v} = (v_x, v_y)$ and $\vv{w} = (w_x, w_y)$, then $\dotv{v}{w} = v_xw_x + v_yw_y$. We can implement it like this:
\begin{lstlisting}
T dot(pt v, pt w) {return v.x*w.x + v.y*w.y;}
\end{lstlisting}

Dot product is often used for testing if two vectors are perpendicular, since we just need to test whether $\dotv{v}{w} = 0$:
\begin{lstlisting}
bool isPerp(pt v, pt w) {return dot(v,w) == 0;}
\end{lstlisting}

It can also be used for finding the angle between two vectors, in $[0,\pi]$. Because of precision errors, we need to be careful not to call \lstinline|acos| with a value that is out of the allowable range $[-1,1]$.
\begin{lstlisting}
double angle(pt v, pt w) {
    double cosTheta = dot(v,w) / abs(v) / abs(w);
    return acos(max(-1.0, min(1.0, cosTheta)));
}
\end{lstlisting}
Since C++17, this can be simplified to:
\begin{lstlisting}
double angle(pt v, pt w) {
    return acos(clamp(dot(v,w) / abs(v) / abs(w), -1.0, 1.0));
}
\end{lstlisting}

\subsection{Cross product}\label{ss:cross}
The cross product $\crossv{v}{w}$ of two vectors $\vv{v}$ and $\vv{w}$ can be seen as a measure of how perpendicular they are. It is defined in 2D as \[\crossv{v}{w} = \normv{v}\normv{w} \sin\theta\] where $\normv{v}$ and $\normv{w}$ are the lengths of the vectors and $\theta$ is amplitude of the oriented angle from $\vv{v}$ to $\vv{w}$.

Since $\sin(-\theta) = -\sin(\theta)$, the sign of the angle matters, the cross product changes sign when the vectors are swapped: $\crossv{w}{v} = -\crossv{v}{w}$. It is positive if $\vv{w}$ is ``to the left'' of $\vv{v}$, and negative is $\vv{w}$ is ``to the right'' of $\vv{v}$.

\centerFig{products2}

In general, we take $\theta$ in $(-\pi,\pi]$, so that the dot product is positive if $0 < \theta < \pi$, negative if $-\pi < \theta < 0$ and zero if $\theta = 0$ or $\theta = \pi$, that is, if $\vv{v}$ and $\vv{w}$ are aligned.

\centerFig{products3}


If we fix $\normv{v}$ and $\normv{w}$ as above, the cross product is maximal when the vectors are perpendicular with $\vv{w}$ on the left, because $\sin\theta = \sin(\pi/2) = 1$, and minimal when they are perpendicular with $\vv{w}$ on the right, because $\sin\theta = \sin(-\pi/2) = -1$.


\begin{mathy}
Because of the definition of sine in right triangles, cross product can also be interpreted in an interesting way (assuming $v \neq 0$): $\crossv{v}{w} = \normv{v}d_{\vv{v}}(\vv{w})$, where $d_{\vv{v}}(\vv{w}) = \normv{w}\sin\theta$ is the \emph{signed} distance from the line that contains $\vv{v}$, with positive values on the left side of $\vv{v}$.
In particular, this means that the cross product doesn't change if one of the vectors moves parallel to the other.

\centerFig{products4}
\end{mathy}

Like dot product, cross product has a very simple expression in cartesian coordinates: if $\vv{v} = (v_x, v_y)$ and $\vv{w} = (w_x, w_y)$, then $\crossv{v}{w} = v_xw_y - v_yw_x$:
\begin{lstlisting}
T cross(pt v, pt w) {return v.x*w.y - v.y*w.x;}
\end{lstlisting}

\begin{trick}
    When using \lstinline|complex|, we can implement both \lstinline|dot()| and \lstinline|cross()| with this trick, which is admittedly quite cryptic, but requires less typing and is less prone to typos:
    \begin{lstlisting}
    T   dot(pt v, pt w) {return (conj(v)*w).x;}
    T cross(pt v, pt w) {return (conj(v)*w).y;}
    \end{lstlisting}

    Here \lstinline|conj()| is the complex conjugate: the conjugate of a complex number $a+bi$ is defined as $a-bi$. To verify that the implementation is correct, we can compute \lstinline|conj(v)*w| as
    \[(v_x-v_yi)\ast(w_x+w_yi) = (v_xw_x + v_yw_y) + (v_xw_y-v_yw_x)i\]
    and see that the real and imaginary parts are indeed the dot product and the cross product.
\end{trick}

\subsubsection{Orientation}
One of the main uses of cross product is in determining the relative position of points and other objects. For this, we define the function $\orient(A,B,C) = \crossv{AB}{AC}$. It is positive if $C$ is on the left side of $\vv{AB}$, negative on the right side, and zero if $C$ is on the line containing $\vv{AB}$. It is straightforward to implement:
\begin{lstlisting}
T orient(pt a, pt b, pt c) {return cross(b-a,c-a);}
\end{lstlisting}

In other words, $\orient(A,B,C)$ is positive if when going from $A$ to $B$ to $C$ we turn left, negative if we turn right, and zero if $A,B,C$ are collinear.

\centerFig{products5}

Its value is conserved by cyclic rotation, that is \[\orient(A,B,C) = \orient(B,C,A) = \orient(C,A,B)\] while swapping any two arguments switches the sign.

As an example of use, suppose we want to check if point $P$ lies in the angle formed by lines $AB$ and $AC$. We can follow this procedure:
\begin{enumerate}
\item check that $\orient(A,B,C) \neq 0$ (otherwise the question is invalid);
\item if $\orient(A,B,C) < 0$, swap $B$ and $C$;
\centerFig{products6}
\item $P$ is in the angle iff $\orient(A,B,P) \geq 0$ and $\orient(A,C,P) \leq 0$.
\centerFig{products7}
\end{enumerate}
\begin{lstlisting}
bool inAngle(pt a, pt b, pt c, pt p) {
    assert(orient(a,b,c) != 0);
    if (orient(a,b,c) < 0) swap(b,c);
    return orient(a,b,p) >= 0 && orient(a,c,p) <= 0;
}
\end{lstlisting}

Using $\orient()$ we can also easily compute the amplitude of an \emph{oriented angle} $\widehat{BAC}$, that is, the angle that is covered if we turn from $B$ to $C$ around $A$ counterclockwise.

There are two cases: either $\orient(A,B,C) \geq 0$, with an angle in $[0,\pi]$, or $\orient(A,B,C) < 0$, with an angle in $(\pi,2\pi)$. In the first case, we can simply use the \lstinline|angle()| function we create based on the dot product; in the second case, we should take ``the other side'', so $2\pi$ minus that result.

\centerFig{products8}

\begin{lstlisting}
double orientedAngle(pt a, pt b, pt c) {
    if (orient(a,b,c) >= 0)
        return angle(b-a, c-a);
    else
        return 2*M_PI - angle(b-a, c-a);
}
\end{lstlisting}

Yet another use case is checking if a polygon $P_1\cdots P_n$ is convex: we compute the $n$ orientations of three consecutive vertices $\orient(P_i,P_{i+1},P_{i+2})$, wrapping around from $n$ to $1$ when necessary. The polygon is convex if they are all $\geq 0$ or all $\leq 0$, depending on the order in which the vertices are given.

\centerFig{products9}

\begin{lstlisting}
bool isConvex(vector<pt> p) {
    bool hasPos=false, hasNeg=false;
    for (int i=0, n=p.size(); i<n; i++) {
        int o = orient(p[i], p[(i+1)%n], p[(i+2)%n]);
        if (o > 0) hasPos = true;
        if (o < 0) hasNeg = true;
    }
    return !(hasPos && hasNeg);
}
\end{lstlisting}

\subsubsection{Polar sort}\label{polar-sort}
Because it can determine whether a vector points to the left or right of another, a common use of cross product is to sort vectors by direction. This is called polar sort: points are sorted in the order that a rotating ray emanating from the origin would touch them. Here, we will try to use cross product to safely sort the points by their arguments in $(-\half,\half]$, that is the order that would be given by the \lstinline|arg()| function for \lstinline|complex|.\footnote{Sorting by using the \lstinline|arg()| value would likely be a bad idea: there is no guarantee (that I know of) that vectors which are multiples of each other will have the same argument, because of precision issues. However, I haven't been to find an example where it fails for values small enough to be handled exactly with \lstinline|long long|.}

\centerFig{products10}

In general $\vv{v}$ should go before $\vv{w}$ when $\crossv{v}{w} > 0$, because that means $\vv{w}$ is to the left of $\vv{v}$ when looking from the origin. This test works well for directions that are sufficiently close: for example, $\crossv{v_2}{v_3} > 0$. But when they are more than $180\degree$ apart in the order, it stops working: for example $\crossv{v_1}{v_5} < 0$. So we first need to split the points in two halves according to their argument:

\centerFig{products11}

If we isolate the points with argument in $(0,\half]$ (region highlighted in blue) from those with argument in $(-\half, 0]$, then the cross product always gives the correct order.
This gives the following algorithm:
\begin{lstlisting}
bool half(pt p) { // true if in blue half
    assert(p.x != 0 || p.y != 0); // the argument of (0,0) is undefined
    return p.y > 0 || (p.y == 0 && p.x < 0);
}
void polarSort(vector<pt> &v) {
    sort(v.begin(), v.end(), [](pt v, pt w) {
        return make_tuple(half(v), 0) <
               make_tuple(half(w), cross(v,w));
    });
}
\end{lstlisting}
Indeed, the comparator will return \lstinline|true| if either $\vv{w}$ is in the blue region and $\vv{v}$ is not, or if they are in the same region and $\crossv{v}{w} > 0$.

We can extend this algorithm in three ways:
\begin{itemize}
    \item Right now, points that are in the exact same direction are considered equal, and thus will be sorted arbitrarily. If we want, we can use their magnitude as a tie breaker:
    \begin{lstlisting}
    void polarSort(vector<pt> &v) {
        sort(v.begin(), v.end(), [](pt v, pt w) {
            return make_tuple(half(v), 0, sq(v)) <
                   make_tuple(half(w), cross(v,w), sq(w));
        });
    }
    \end{lstlisting}
    With this tweak, if two points are in the same direction, the point that is further from the origin will appear later.
    \item We can perform a polar sort around some point $O$ other than the origin: we just have to subtract that point $O$ from the vectors $\vv{v}$ and $\vv{w}$ when comparing them. This as if we translated the whole plane so that $O$ is moved to $(0,0)$:
    \begin{lstlisting}
    void polarSortAround(pt o, vector<pt> &v) {
        sort(v.begin(), v.end(), [](pt v, pt w) {
            return make_tuple(half(v-o), 0)) <
                   make_tuple(half(w-o), cross(v-o, w-o));
        });
    }
    \end{lstlisting}
    \item Finally, the starting angle of the ordering can be modified easily by tweaking function \lstinline|half()|. For example, if we want some vector $\vv{v}$ to be the first angle in the polar sort, we can write:
    \begin{lstlisting}
    pt v = {/* whatever you want except 0,0 */};
    bool half(pt p) {
        return cross(v,p) < 0 || (cross(v,p) == 0 && dot(v,p) < 0);
    }
    \end{lstlisting}
    This places the blue region like this:
    \centerFig{products12}
\end{itemize}
